\begin{question}[2]
Which experiment provided direct evidence for the particle nature of light?

\begin{oneparcheckboxes}
\choice Young's double-slit experiment
\choice Michelson-Morley experiment
\correctchoice Compton effect
\choice Davisson-Germer experiment
\end{oneparcheckboxes}
\end{question}

\begin{question}[2]
The Compton wavelength shift is given by:

\begin{oneparcheckboxes}
\correctchoice $\displaystyle \Delta\lambda = \frac{h}{m_ec}(1-\cos\theta)$
\choice $\displaystyle \Delta\lambda = \frac{h}{m_ec}(1+\cos\theta)$
\choice $\displaystyle \Delta\lambda = \frac{h}{m_ec}\sin\theta$
\choice $\displaystyle \Delta\lambda = \frac{h}{m_ec}\tan\theta$
\end{oneparcheckboxes}
\end{question}

\begin{question}[2]
According to de Broglie's hypothesis, the wavelength of a particle is:

\begin{oneparcheckboxes}
\choice Directly proportional to its momentum
\correctchoice Inversely proportional to its momentum
\choice Directly proportional to its energy
\choice Inversely proportional to its energy
\end{oneparcheckboxes}
\end{question}

\begin{question}[2]
The Davisson-Germer experiment demonstrated:

\begin{oneparcheckboxes}
\choice Wave nature of light
\choice Particle nature of light
\correctchoice Wave nature of electrons
\choice Particle nature of electrons
\end{oneparcheckboxes}
\end{question}

\begin{question}[2]
Heisenberg's Uncertainty Principle states that:

\begin{oneparcheckboxes}
\correctchoice $\displaystyle \Delta x \Delta p \geq \frac{h}{4\pi}$
\choice $\displaystyle \Delta x \Delta p \leq \frac{h}{4\pi}$
\choice $\displaystyle \Delta x \Delta p = \frac{h}{4\pi}$
\choice $\displaystyle \Delta x \Delta p = h$
\end{oneparcheckboxes}
\end{question}

\begin{question}[2]
The time-independent Schrödinger equation is:

\begin{oneparcheckboxes}
\correctchoice $\displaystyle \hat{H}\psi = E\psi$
\choice $\displaystyle \hat{H}\psi = i\hbar\frac{\partial\psi}{\partial t}$
\choice $\displaystyle \nabla^2\psi + k^2\psi = 0$
\choice $\displaystyle \frac{\partial^2\psi}{\partial x^2} = \frac{1}{v^2}\frac{\partial^2\psi}{\partial t^2}$
\end{oneparcheckboxes}
\end{question}

\begin{question}[2]
The quantization of energy in a hydrogen atom is given by:

\begin{oneparcheckboxes}
\correctchoice $\displaystyle E_n = -\frac{13.6}{n^2}$ eV
\choice $\displaystyle E_n = \frac{13.6}{n^2}$ eV
\choice $\displaystyle E_n = -\frac{13.6}{n}$ eV
\choice $\displaystyle E_n = \frac{13.6}{n}$ eV
\end{oneparcheckboxes}
\end{question}

\begin{question}[2]
The wave function $\psi(x)$ must satisfy:

\begin{oneparcheckboxes}
\choice $\psi(x)$ must be real
\choice $\psi(x)$ must be positive
\correctchoice $\psi(x)$ must be single-valued
\choice $\psi(x)$ must be constant
\end{oneparcheckboxes}
\end{question}

\begin{question}[2]
The probability density is given by:

\begin{oneparcheckboxes}
\choice $\displaystyle |\psi(x)|$
\correctchoice $\displaystyle |\psi(x)|^2$
\choice $\displaystyle \psi(x)^2$
\choice $\displaystyle \psi(x)$
\end{oneparcheckboxes}
\end{question}

\begin{question}[2]
The energy levels of a particle in a one-dimensional box are:

\begin{oneparcheckboxes}
\choice $E_n = \frac{n^2h^2}{8mL}$
\correctchoice $E_n = \frac{n^2h^2}{8mL^2}$
\choice $E_n = \frac{nh^2}{8mL^2}$
\choice $E_n = \frac{n^2h}{8mL^2}$
\end{oneparcheckboxes}
\end{question}

\begin{question}[2]
The ground state energy of a harmonic oscillator is:

\begin{oneparcheckboxes}
\choice 0
\correctchoice $\frac{1}{2}\hbar\omega$
\choice $\hbar\omega$
\choice $\frac{3}{2}\hbar\omega$
\end{oneparcheckboxes}
\end{question}

\begin{question}[2]
The Pauli exclusion principle states that:

\begin{oneparcheckboxes}
\choice No two electrons can have the same energy
\correctchoice No two electrons can have the same set of quantum numbers
\choice No two electrons can be in the same orbital
\choice No two electrons can have the same spin
\end{oneparcheckboxes}
\end{question}

\begin{question}[2]
The spin quantum number can take values:

\begin{oneparcheckboxes}
\choice 0, 1, 2, ...
\choice 0, $\frac{1}{2}$, 1, $\frac{3}{2}$, ...
\correctchoice $\pm\frac{1}{2}$
\choice Any real number
\end{oneparcheckboxes}
\end{question}

\begin{question}[2]
The orbital angular momentum quantum number l can take values:

\begin{oneparcheckboxes}
\choice 0, 1, 2, ..., n
\correctchoice 0, 1, 2, ..., n-1
\choice 1, 2, 3, ..., n
\choice 0, $\frac{1}{2}$, 1, $\frac{3}{2}$, ..., n
\end{oneparcheckboxes}
\end{question}

\begin{question}[2]
The magnetic quantum number m can take values:

\begin{oneparcheckboxes}
\choice 0, 1, 2, ..., l
\choice 0, $\pm1$, $\pm2$, ..., $\pm l$
\correctchoice -l, -l+1, ..., 0, ..., l-1, l
\choice Any integer
\end{oneparcheckboxes}
\end{question}

\begin{question}[2]
The principal quantum number n can take values:

\begin{oneparcheckboxes}
\choice 0, 1, 2, ...
\correctchoice 1, 2, 3, ...
\choice $\frac{1}{2}$, 1, $\frac{3}{2}$, 2, ...
\choice Any positive number
\end{oneparcheckboxes}
\end{question} 