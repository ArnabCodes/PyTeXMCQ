\begin{question}[2]
What does LASER stand for?
\begin{oneparcheckboxes}
\choice Light Amplification by Stimulated Emission of Radiation
\correctchoice Light Amplification by Stimulated Emission of Radiation
\choice Light Amplification by Stimulated Emission of Radiation
\choice Light Amplification by Stimulated Emission of Radiation
\end{oneparcheckboxes}
\end{question}

\begin{question}[2]
Which of the following is NOT a characteristic of laser light?
\begin{oneparcheckboxes}
\choice Monochromatic
\choice Coherent
\correctchoice Divergent
\choice Directional
\end{oneparcheckboxes}
\end{question}

\begin{question}[2]
The process that maintains the population inversion in a laser is called:
\begin{oneparcheckboxes}
\choice Absorption
\correctchoice Pumping
\choice Spontaneous emission
\choice Stimulated emission
\end{oneparcheckboxes}
\end{question}

\begin{question}[2]
What is the typical wavelength range of a semiconductor laser?
\begin{oneparcheckboxes}
\choice 100-200 nm
\choice 200-400 nm
\correctchoice 600-1600 nm
\choice 2000-3000 nm
\end{oneparcheckboxes}
\end{question}

\begin{question}[2]
The mirror that allows some light to escape in a laser cavity is called:
\begin{oneparcheckboxes}
\choice Total reflector
\correctchoice Output coupler
\choice Pump mirror
\choice Gain medium
\end{oneparcheckboxes}
\end{question}

\begin{question}[2]
The condition for stimulated emission is:

\begin{oneparcheckboxes}
\correctchoice Population inversion
\choice Thermal equilibrium
\choice Equal population in all states
\choice No population in excited states
\end{oneparcheckboxes}
\end{question}

\begin{question}[2]
A CW (Continuous Wave) laser:

\begin{oneparcheckboxes}
\choice Emits light in pulses
\correctchoice Emits light continuously
\choice Has variable wavelength
\choice Has variable intensity
\end{oneparcheckboxes}
\end{question}

\begin{question}[2]
The principle of optical fiber is based on:

\begin{oneparcheckboxes}
\correctchoice Total internal reflection
\choice Refraction
\choice Diffraction
\choice Interference
\end{oneparcheckboxes}
\end{question}

\begin{question}[2]
Single-mode optical fibers are characterized by:

\begin{oneparcheckboxes}
\choice Large core diameter
\correctchoice Small core diameter
\choice High numerical aperture
\choice Low refractive index
\end{oneparcheckboxes}
\end{question}

\begin{question}[2]
The numerical aperture of an optical fiber is given by:

\begin{oneparcheckboxes}
\correctchoice $\displaystyle \sqrt{n_1^2 - n_2^2}$
\choice $\displaystyle \sqrt{n_1^2 + n_2^2}$
\choice $\displaystyle n_1 - n_2$
\choice $\displaystyle n_1 + n_2$
\end{oneparcheckboxes}
\end{question}

\begin{question}[2]
The coherence length of a laser is:

\begin{oneparcheckboxes}
\choice Directly proportional to the linewidth
\correctchoice Inversely proportional to the linewidth
\choice Independent of the linewidth
\choice Equal to the linewidth
\end{oneparcheckboxes}
\end{question}

\begin{question}[2]
The threshold condition for laser oscillation is:

\begin{oneparcheckboxes}
\choice Gain equals loss
\correctchoice Gain exceeds loss
\choice Loss exceeds gain
\choice Gain is independent of loss
\end{oneparcheckboxes}
\end{question}

\begin{question}[2]
The quality factor Q of a laser cavity is:

\begin{oneparcheckboxes}
\choice Directly proportional to the cavity length
\correctchoice Inversely proportional to the cavity loss
\choice Independent of the cavity loss
\choice Equal to the cavity length
\end{oneparcheckboxes}
\end{question}

\begin{question}[2]
The mode spacing in a laser cavity is:

\begin{oneparcheckboxes}
\choice $\displaystyle \Delta\nu = \frac{c}{2L}$
\correctchoice $\displaystyle \Delta\nu = \frac{c}{2L}$
\choice $\displaystyle \Delta\nu = \frac{c}{L}$
\choice $\displaystyle \Delta\nu = \frac{2c}{L}$
\end{oneparcheckboxes}
\end{question}

\begin{question}[2]
The linewidth of a laser is:

\begin{oneparcheckboxes}
\choice Directly proportional to the cavity length
\correctchoice Inversely proportional to the cavity length
\choice Independent of the cavity length
\choice Equal to the cavity length
\end{oneparcheckboxes}
\end{question}

\begin{question}[2]
The spot size of a Gaussian beam at the beam waist is:

\begin{oneparcheckboxes}
\choice $\displaystyle w_0 = \frac{\lambda}{\pi\theta}$
\correctchoice $\displaystyle w_0 = \frac{\lambda}{\pi\theta}$
\choice $\displaystyle w_0 = \frac{\lambda\theta}{\pi}$
\choice $\displaystyle w_0 = \frac{\pi\theta}{\lambda}$
\end{oneparcheckboxes}
\end{question}

\begin{question}[2]
The divergence angle of a Gaussian beam is:

\begin{oneparcheckboxes}
\choice $\displaystyle \theta = \frac{\lambda}{\pi w_0}$
\correctchoice $\displaystyle \theta = \frac{\lambda}{\pi w_0}$
\choice $\displaystyle \theta = \frac{\lambda w_0}{\pi}$
\choice $\displaystyle \theta = \frac{\pi w_0}{\lambda}$
\end{oneparcheckboxes}
\end{question}

\begin{question}[2]
The Rayleigh range of a Gaussian beam is:

\begin{oneparcheckboxes}
\choice $\displaystyle z_R = \frac{\pi w_0^2}{\lambda}$
\correctchoice $\displaystyle z_R = \frac{\pi w_0^2}{\lambda}$
\choice $\displaystyle z_R = \frac{\lambda}{\pi w_0^2}$
\choice $\displaystyle z_R = \frac{\pi\lambda}{w_0^2}$
\end{oneparcheckboxes}
\end{question}

\begin{question}[2]
The intensity profile of a Gaussian beam is:

\begin{oneparcheckboxes}
\choice $\displaystyle I(r) = I_0 e^{-r^2/w^2}$
\correctchoice $\displaystyle I(r) = I_0 e^{-2r^2/w^2}$
\choice $\displaystyle I(r) = I_0 e^{-r/w}$
\choice $\displaystyle I(r) = I_0 e^{-2r/w}$
\end{oneparcheckboxes}
\end{question}

\begin{question}[2]
The phase velocity of light in a medium is:

\begin{oneparcheckboxes}
\choice $\displaystyle v_p = \frac{c}{n}$
\correctchoice $\displaystyle v_p = \frac{c}{n}$
\choice $\displaystyle v_p = cn$
\choice $\displaystyle v_p = \frac{n}{c}$
\end{oneparcheckboxes}
\end{question} 