\begin{question}[2]
For a particle in a 1D infinite potential well of width L, the energy levels are given by:

\begin{oneparcheckboxes}
\correctchoice $\displaystyle E_n = \frac{n^2h^2}{8mL^2}$
\choice $\displaystyle E_n = \frac{n^2h^2}{2mL^2}$
\choice $\displaystyle E_n = \frac{nh^2}{8mL^2}$
\choice $\displaystyle E_n = \frac{nh^2}{2mL^2}$
\end{oneparcheckboxes}
\end{question}

\begin{question}[2]
The ground state energy of a particle in a 3D infinite potential well is:

\begin{oneparcheckboxes}
\correctchoice $\displaystyle \frac{3h^2}{8mL^2}$
\choice $\displaystyle \frac{h^2}{8mL^2}$
\choice $\displaystyle \frac{3h^2}{2mL^2}$
\choice $\displaystyle \frac{h^2}{2mL^2}$
\end{oneparcheckboxes}
\end{question}

\begin{question}[2]
Which of the following is NOT a quantum mechanical operator?

\begin{oneparcheckboxes}
\choice Position operator $\displaystyle \hat{x}$
\choice Momentum operator $\displaystyle -i\hbar\frac{\partial}{\partial x}$
\choice Energy operator $\displaystyle i\hbar\frac{\partial}{\partial t}$
\correctchoice Velocity operator $\displaystyle \frac{dx}{dt}$
\end{oneparcheckboxes}
\end{question}

\begin{question}[2]
Eigenstates of an operator are:

\begin{oneparcheckboxes}
\choice States that are not affected by the operator
\correctchoice States that are multiplied by a constant when operated upon
\choice States that are orthogonal to each other
\choice States that have zero energy
\end{oneparcheckboxes}
\end{question}

\begin{question}[2]
In quantum computing, a qubit can be in:

\begin{oneparcheckboxes}
\choice Only 0 or 1 state
\correctchoice Any superposition of 0 and 1 states
\choice Only 0 state
\choice Only 1 state
\end{oneparcheckboxes}
\end{question}

\begin{question}[2]
The Pauli matrices are used in quantum computing to represent:

\begin{oneparcheckboxes}
\correctchoice Single-qubit gates
\choice Two-qubit gates
\choice Measurement operations
\choice Quantum error correction
\end{oneparcheckboxes}
\end{question}

\begin{question}[2]
The quantum tunneling effect is most significant when:

\begin{oneparcheckboxes}
\choice The barrier is very wide
\correctchoice The barrier is very narrow
\choice The particle energy is very high
\choice The particle mass is very large
\end{oneparcheckboxes}
\end{question}

\begin{question}[2]
The quantum harmonic oscillator has energy levels:

\begin{oneparcheckboxes}
\choice $\displaystyle E_n = n\hbar\omega$
\correctchoice $\displaystyle E_n = (n + \frac{1}{2})\hbar\omega$
\choice $\displaystyle E_n = n^2\hbar\omega$
\choice $\displaystyle E_n = \frac{1}{2}n\hbar\omega$
\end{oneparcheckboxes}
\end{question}

\begin{question}[2]
The quantum mechanical tunneling probability depends on:

\begin{oneparcheckboxes}
\choice Only the barrier height
\choice Only the barrier width
\correctchoice Both barrier height and width
\choice Neither barrier height nor width
\end{oneparcheckboxes}
\end{question}

\begin{question}[2]
The quantum mechanical wave function for a free particle is:

\begin{oneparcheckboxes}
\choice A standing wave
\correctchoice A plane wave
\choice A spherical wave
\choice A Gaussian wave packet
\end{oneparcheckboxes}
\end{question}

\begin{question}[2]
The quantum mechanical expectation value of an observable A is given by:

\begin{oneparcheckboxes}
\choice $\displaystyle \langle A \rangle = \int \psi^* A \psi dx$
\correctchoice $\displaystyle \langle A \rangle = \int \psi^* A \psi dx$
\choice $\displaystyle \langle A \rangle = \int \psi A \psi dx$
\choice $\displaystyle \langle A \rangle = \int \psi^* A^* \psi dx$
\end{oneparcheckboxes}
\end{question}

\begin{question}[2]
The quantum mechanical probability current density is given by:

\begin{oneparcheckboxes}
\choice $\displaystyle j = \frac{\hbar}{2mi}(\psi^*\nabla\psi - \psi\nabla\psi^*)$
\correctchoice $\displaystyle j = \frac{\hbar}{2mi}(\psi^*\nabla\psi - \psi\nabla\psi^*)$
\choice $\displaystyle j = \frac{\hbar}{mi}(\psi^*\nabla\psi)$
\choice $\displaystyle j = \frac{\hbar}{mi}(\psi\nabla\psi^*)$
\end{oneparcheckboxes}
\end{question}

\begin{question}[2]
The quantum mechanical momentum operator in position space is:

\begin{oneparcheckboxes}
\choice $\displaystyle \hat{p} = -i\hbar\frac{\partial}{\partial x}$
\correctchoice $\displaystyle \hat{p} = -i\hbar\frac{\partial}{\partial x}$
\choice $\displaystyle \hat{p} = i\hbar\frac{\partial}{\partial x}$
\choice $\displaystyle \hat{p} = \hbar\frac{\partial}{\partial x}$
\end{oneparcheckboxes}
\end{question}

\begin{question}[2]
The quantum mechanical position operator in momentum space is:

\begin{oneparcheckboxes}
\choice $\displaystyle \hat{x} = i\hbar\frac{\partial}{\partial p}$
\correctchoice $\displaystyle \hat{x} = i\hbar\frac{\partial}{\partial p}$
\choice $\displaystyle \hat{x} = -i\hbar\frac{\partial}{\partial p}$
\choice $\displaystyle \hat{x} = \hbar\frac{\partial}{\partial p}$
\end{oneparcheckboxes}
\end{question}

\begin{question}[2]
The quantum mechanical Hamiltonian operator for a free particle is:

\begin{oneparcheckboxes}
\choice $\displaystyle \hat{H} = \frac{\hat{p}^2}{2m}$
\correctchoice $\displaystyle \hat{H} = \frac{\hat{p}^2}{2m}$
\choice $\displaystyle \hat{H} = \frac{\hat{p}}{2m}$
\choice $\displaystyle \hat{H} = \frac{\hat{p}^2}{m}$
\end{oneparcheckboxes}
\end{question}

\begin{question}[2]
The quantum mechanical time evolution operator is:

\begin{oneparcheckboxes}
\choice $\displaystyle \hat{U}(t) = e^{-i\hat{H}t/\hbar}$
\correctchoice $\displaystyle \hat{U}(t) = e^{-i\hat{H}t/\hbar}$
\choice $\displaystyle \hat{U}(t) = e^{i\hat{H}t/\hbar}$
\choice $\displaystyle \hat{U}(t) = e^{-\hat{H}t/\hbar}$
\end{oneparcheckboxes}
\end{question} 