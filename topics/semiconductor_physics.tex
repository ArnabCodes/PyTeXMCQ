\begin{question}[2]
Energy bands in solids are formed due to:

\begin{oneparcheckboxes}
\correctchoice Pauli's exclusion principle
\choice Heisenberg's uncertainty principle
\choice Wave-particle duality
\choice Quantum tunneling
\end{oneparcheckboxes}
\end{question}

\begin{question}[2]
An intrinsic semiconductor at absolute zero temperature:

\begin{oneparcheckboxes}
\correctchoice Has no free electrons
\choice Has maximum conductivity
\choice Has equal number of electrons and holes
\choice Has maximum resistance
\end{oneparcheckboxes}
\end{question}

\begin{question}[2]
In an n-type semiconductor, the Fermi level:

\begin{oneparcheckboxes}
\choice Lies in the middle of the band gap
\correctchoice Moves towards the conduction band
\choice Moves towards the valence band
\choice Lies at the top of the valence band
\end{oneparcheckboxes}
\end{question}

\begin{question}[2]
The Hall effect can be used to determine:

\begin{oneparcheckboxes}
\correctchoice The type of charge carriers
\choice The band gap of the material
\choice The dielectric constant
\choice The refractive index
\end{oneparcheckboxes}
\end{question}

\begin{question}[2]
In a p-n junction diode, the depletion region is formed due to:

\begin{oneparcheckboxes}
\correctchoice Diffusion of charge carriers
\choice Drift of charge carriers
\choice Thermal generation
\choice Impact ionization
\end{oneparcheckboxes}
\end{question}

\begin{question}[2]
The current gain in a common-emitter transistor configuration is:

\begin{oneparcheckboxes}
\choice $\displaystyle \alpha$
\correctchoice $\displaystyle \beta$
\choice $\displaystyle \gamma$
\choice $\displaystyle \delta$
\end{oneparcheckboxes}
\end{question}

\begin{question}[2]
The mobility of charge carriers in a semiconductor:

\begin{oneparcheckboxes}
\choice Increases with temperature
\correctchoice Decreases with temperature
\choice Is independent of temperature
\choice First increases then decreases with temperature
\end{oneparcheckboxes}
\end{question}

\begin{question}[2]
The effective mass of an electron in a semiconductor:

\begin{oneparcheckboxes}
\choice Is equal to the rest mass of electron
\correctchoice Is different from the rest mass of electron
\choice Is zero
\choice Is infinite
\end{oneparcheckboxes}
\end{question}

\begin{question}[2]
The intrinsic carrier concentration in a semiconductor:

\begin{oneparcheckboxes}
\choice Decreases with temperature
\correctchoice Increases with temperature
\choice Is independent of temperature
\choice First increases then decreases with temperature
\end{oneparcheckboxes}
\end{question}

\begin{question}[2]
The band gap of a semiconductor:

\begin{oneparcheckboxes}
\choice Increases with temperature
\correctchoice Decreases with temperature
\choice Is independent of temperature
\choice First increases then decreases with temperature
\end{oneparcheckboxes}
\end{question}

\begin{question}[2]
The diffusion length of minority carriers:

\begin{oneparcheckboxes}
\choice Is the distance traveled in one second
\correctchoice Is the average distance before recombination
\choice Is the distance between electrodes
\choice Is the width of the depletion region
\end{oneparcheckboxes}
\end{question}

\begin{question}[2]
The built-in potential in a p-n junction:

\begin{oneparcheckboxes}
\choice Increases with temperature
\correctchoice Decreases with temperature
\choice Is independent of temperature
\choice First increases then decreases with temperature
\end{oneparcheckboxes}
\end{question}

\begin{question}[2]
The reverse saturation current in a p-n junction:

\begin{oneparcheckboxes}
\choice Decreases with temperature
\correctchoice Increases with temperature
\choice Is independent of temperature
\choice First increases then decreases with temperature
\end{oneparcheckboxes}
\end{question}

\begin{question}[2]
The capacitance of a p-n junction:

\begin{oneparcheckboxes}
\choice Increases with reverse bias
\correctchoice Decreases with reverse bias
\choice Is independent of bias
\choice First increases then decreases with bias
\end{oneparcheckboxes}
\end{question}

\begin{question}[2]
The breakdown voltage of a p-n junction:

\begin{oneparcheckboxes}
\choice Increases with doping concentration
\correctchoice Decreases with doping concentration
\choice Is independent of doping concentration
\choice First increases then decreases with doping concentration
\end{oneparcheckboxes}
\end{question}

\begin{question}[2]
The transit time of carriers in a semiconductor:

\begin{oneparcheckboxes}
\choice Is the time taken to cross the device
\correctchoice Is the time taken to reach the opposite electrode
\choice Is the time between collisions
\choice Is the time between generation and recombination
\end{oneparcheckboxes}
\end{question}

\begin{question}[2]
What is the energy gap of silicon at room temperature?
\begin{oneparcheckboxes}
\choice 0.5 eV
\correctchoice 1.1 eV
\choice 1.5 eV
\choice 2.0 eV
\end{oneparcheckboxes}
\end{question}

\begin{question}[2]
Which of the following is a direct bandgap semiconductor?
\begin{oneparcheckboxes}
\choice Silicon
\choice Germanium
\correctchoice Gallium Arsenide
\choice Silicon Carbide
\end{oneparcheckboxes}
\end{question}

\begin{question}[2]
The process of adding impurities to a semiconductor is called:
\begin{oneparcheckboxes}
\choice Oxidation
\correctchoice Doping
\choice Diffusion
\choice Implantation
\end{oneparcheckboxes}
\end{question}

\begin{question}[2]
What type of semiconductor is created when phosphorus is added to silicon?
\begin{oneparcheckboxes}
\choice p-type
\correctchoice n-type
\choice Intrinsic
\choice Compensated
\end{oneparcheckboxes}
\end{question}

\begin{question}[2]
The majority carriers in a p-type semiconductor are:
\begin{oneparcheckboxes}
\choice Electrons
\correctchoice Holes
\choice Both electrons and holes
\choice Neither electrons nor holes
\end{oneparcheckboxes}
\end{question} 