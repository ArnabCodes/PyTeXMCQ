\begin{question}[2]
The number of atoms per unit cell in a simple cubic (SC) structure is:

\begin{oneparcheckboxes}
\choice 1
\choice 2
\correctchoice 4
\choice 8
\end{oneparcheckboxes}
\end{question}

\begin{question}[2]
The coordination number for a body-centered cubic (BCC) structure is:

\begin{oneparcheckboxes}
\choice 6
\choice 8
\choice 12
\correctchoice 14
\end{oneparcheckboxes}
\end{question}

\begin{question}[2]
In a face-centered cubic (FCC) structure, the packing efficiency is:

\begin{oneparcheckboxes}
\choice 52\%
\choice 68\%
\correctchoice 74\%
\choice 82\%
\end{oneparcheckboxes}
\end{question}

\begin{question}[2]
Miller indices (hkl) represent:

\begin{oneparcheckboxes}
\choice A point in the crystal
\choice A direction in the crystal
\correctchoice A plane in the crystal
\choice A unit cell
\end{oneparcheckboxes}
\end{question}

\begin{question}[2]
Bragg's law is given by:

\begin{oneparcheckboxes}
\correctchoice $\displaystyle n\lambda = 2d\sin\theta$
\choice $\displaystyle n\lambda = d\sin\theta$
\choice $\displaystyle \lambda = 2d\sin\theta$
\choice $\displaystyle \lambda = d\sin\theta$
\end{oneparcheckboxes}
\end{question}

\begin{question}[2]
The interplanar spacing d for a cubic crystal is given by:

\begin{oneparcheckboxes}
\correctchoice $\displaystyle d = \frac{a}{\sqrt{h^2 + k^2 + l^2}}$
\choice $\displaystyle d = \frac{a}{h + k + l}$
\choice $\displaystyle d = a\sqrt{h^2 + k^2 + l^2}$
\choice $\displaystyle d = a(h + k + l)$
\end{oneparcheckboxes}
\end{question}

\begin{question}[2]
The atomic radius in a simple cubic structure is related to the lattice constant by:

\begin{oneparcheckboxes}
\choice $\displaystyle r = a$
\correctchoice $\displaystyle r = \frac{a}{2}$
\choice $\displaystyle r = \frac{a\sqrt{2}}{4}$
\choice $\displaystyle r = \frac{a\sqrt{3}}{4}$
\end{oneparcheckboxes}
\end{question}

\begin{question}[2]
The number of octahedral voids per atom in a close-packed structure is:

\begin{oneparcheckboxes}
\choice 1
\correctchoice 2
\choice 3
\choice 4
\end{oneparcheckboxes}
\end{question}

\begin{question}[2]
The coordination number in a hexagonal close-packed (HCP) structure is:

\begin{oneparcheckboxes}
\choice 6
\choice 8
\correctchoice 12
\choice 14
\end{oneparcheckboxes}
\end{question}

\begin{question}[2]
The packing efficiency of a body-centered cubic (BCC) structure is approximately:

\begin{oneparcheckboxes}
\choice 52\%
\correctchoice 68\%
\choice 74\%
\choice 82\%
\end{oneparcheckboxes}
\end{question}

\begin{question}[2]
The number of tetrahedral voids per atom in a close-packed structure is:

\begin{oneparcheckboxes}
\choice 1
\correctchoice 2
\choice 3
\choice 4
\end{oneparcheckboxes}
\end{question}

\begin{question}[2]
The relationship between the lattice constant 'a' and the atomic radius 'r' in a BCC structure is:

\begin{oneparcheckboxes}
\choice $\displaystyle a = 2r$
\choice $\displaystyle a = 2\sqrt{2}r$
\correctchoice $\displaystyle a = \frac{4r}{\sqrt{3}}$
\choice $\displaystyle a = 4r$
\end{oneparcheckboxes}
\end{question}

\begin{question}[2]
The number of atoms per unit cell in a hexagonal close-packed (HCP) structure is:

\begin{oneparcheckboxes}
\choice 2
\correctchoice 6
\choice 8
\choice 12
\end{oneparcheckboxes}
\end{question}

\begin{question}[2]
The relationship between the lattice constant 'a' and the atomic radius 'r' in an FCC structure is:

\begin{oneparcheckboxes}
\choice $\displaystyle a = 2r$
\correctchoice $\displaystyle a = 2\sqrt{2}r$
\choice $\displaystyle a = \frac{4r}{\sqrt{3}}$
\choice $\displaystyle a = 4r$
\end{oneparcheckboxes}
\end{question}

\begin{question}[2]
The number of atoms per unit cell in a diamond cubic structure is:

\begin{oneparcheckboxes}
\choice 4
\choice 6
\choice 8
\correctchoice 12
\end{oneparcheckboxes}
\end{question}

\begin{question}[2]
The coordination number in a diamond cubic structure is:

\begin{oneparcheckboxes}
\choice 4
\correctchoice 6
\choice 8
\choice 12
\end{oneparcheckboxes}
\end{question} 