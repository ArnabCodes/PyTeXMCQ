\begin{question}[2]
A Zener diode operates in:

\begin{oneparcheckboxes}
\choice Forward bias only
\correctchoice Reverse bias only
\choice Both forward and reverse bias
\choice No bias condition
\end{oneparcheckboxes}
\end{question}

\begin{question}[2]
A photodiode converts:

\begin{oneparcheckboxes}
\correctchoice Light energy to electrical energy
\choice Electrical energy to light energy
\choice Heat energy to electrical energy
\choice Electrical energy to heat energy
\end{oneparcheckboxes}
\end{question}

\begin{question}[2]
The color of light emitted by an LED depends on:

\begin{oneparcheckboxes}
\choice The forward current
\choice The reverse voltage
\correctchoice The band gap of the material
\choice The size of the device
\end{oneparcheckboxes}
\end{question}

\begin{question}[2]
In a BJT, the base current is:

\begin{oneparcheckboxes}
\choice Much larger than collector current
\correctchoice Much smaller than collector current
\choice Equal to collector current
\choice Equal to emitter current
\end{oneparcheckboxes}
\end{question}

\begin{question}[2]
A FET is a:

\begin{oneparcheckboxes}
\correctchoice Voltage-controlled device
\choice Current-controlled device
\choice Power-controlled device
\choice Frequency-controlled device
\end{oneparcheckboxes}
\end{question}

\begin{question}[2]
In a MOSFET, the gate is insulated from the channel by:

\begin{oneparcheckboxes}
\choice A p-n junction
\choice A metal layer
\correctchoice An oxide layer
\choice A vacuum
\end{oneparcheckboxes}
\end{question}

\begin{question}[2]
The pinch-off voltage in a JFET is:

\begin{oneparcheckboxes}
\choice The voltage at which the channel opens
\correctchoice The voltage at which the channel closes
\choice The voltage at which the device breaks down
\choice The voltage at which the device saturates
\end{oneparcheckboxes}
\end{question}

\begin{question}[2]
The Early effect in a BJT is due to:

\begin{oneparcheckboxes}
\choice Base width modulation
\correctchoice Base width modulation
\choice Emitter width modulation
\choice Collector width modulation
\end{oneparcheckboxes}
\end{question}

\begin{question}[2]
The threshold voltage of a MOSFET is:

\begin{oneparcheckboxes}
\choice The voltage at which the device turns off
\correctchoice The voltage at which the device turns on
\choice The voltage at which the device breaks down
\choice The voltage at which the device saturates
\end{oneparcheckboxes}
\end{question}

\begin{question}[2]
The transconductance of a FET is:

\begin{oneparcheckboxes}
\choice $\displaystyle \frac{\Delta I_D}{\Delta V_{DS}}$
\correctchoice $\displaystyle \frac{\Delta I_D}{\Delta V_{GS}}$
\choice $\displaystyle \frac{\Delta V_{DS}}{\Delta I_D}$
\choice $\displaystyle \frac{\Delta V_{GS}}{\Delta I_D}$
\end{oneparcheckboxes}
\end{question}

\begin{question}[2]
The current gain of a BJT in common-emitter configuration is:

\begin{oneparcheckboxes}
\choice $\displaystyle \alpha$
\correctchoice $\displaystyle \beta$
\choice $\displaystyle \gamma$
\choice $\displaystyle \delta$
\end{oneparcheckboxes}
\end{question}

\begin{question}[2]
The input impedance of a common-emitter amplifier is:

\begin{oneparcheckboxes}
\choice Very high
\correctchoice Moderate
\choice Very low
\choice Zero
\end{oneparcheckboxes}
\end{question}

\begin{question}[2]
The output impedance of a common-collector amplifier is:

\begin{oneparcheckboxes}
\choice Very high
\correctchoice Very low
\choice Moderate
\choice Zero
\end{oneparcheckboxes}
\end{question}

\begin{question}[2]
The voltage gain of a common-base amplifier is:

\begin{oneparcheckboxes}
\choice Less than unity
\correctchoice Greater than unity
\choice Equal to unity
\choice Zero
\end{oneparcheckboxes}
\end{question}

\begin{question}[2]
The Miller effect in amplifiers is due to:

\begin{oneparcheckboxes}
\choice Input capacitance
\correctchoice Feedback capacitance
\choice Output capacitance
\choice Parasitic capacitance
\end{oneparcheckboxes}
\end{question}

\begin{question}[2]
The unity-gain bandwidth of an amplifier is:

\begin{oneparcheckboxes}
\choice The frequency at which gain is maximum
\correctchoice The frequency at which gain is unity
\choice The frequency at which gain is minimum
\choice The frequency at which gain is zero
\end{oneparcheckboxes}
\end{question}

\begin{question}[2]
What is the main function of a p-n junction?
\begin{oneparcheckboxes}
\choice To store charge
\correctchoice To allow current flow in one direction
\choice To amplify signals
\choice To store data
\end{oneparcheckboxes}
\end{question}

\begin{question}[2]
In a MOSFET, the gate is insulated from the channel by:
\begin{oneparcheckboxes}
\choice A metal layer
\choice A semiconductor layer
\correctchoice An oxide layer
\choice A polymer layer
\end{oneparcheckboxes}
\end{question}

\begin{question}[2]
The region in a transistor where the current is controlled is called:
\begin{oneparcheckboxes}
\choice Base
\correctchoice Channel
\choice Drain
\choice Source
\end{oneparcheckboxes}
\end{question}

\begin{question}[2]
What happens to the resistance of a thermistor when temperature increases?
\begin{oneparcheckboxes}
\choice Increases
\correctchoice Decreases
\choice Stays the same
\choice Becomes infinite
\end{oneparcheckboxes}
\end{question}

\begin{question}[2]
The process of creating a thin oxide layer on silicon is called:
\begin{oneparcheckboxes}
\choice Doping
\correctchoice Oxidation
\choice Diffusion
\choice Implantation
\end{oneparcheckboxes}
\end{question} 